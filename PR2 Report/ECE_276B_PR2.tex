\documentclass[journal]{IEEEtran} % use the `journal` option for ITherm conference style
\IEEEoverridecommandlockouts
% The preceding line is only needed to identify funding in the first footnote. If that is unneeded, please comment it out.
\usepackage{cite}
% \usepackage{caption}
\usepackage{float}
% \usepackage{nicefrac}
\usepackage{amsmath,amssymb,amsfonts}
\usepackage{nicefrac}
\usepackage{algorithm}
\usepackage{algorithmic}
\usepackage{algpseudocode}
\usepackage{graphicx}
\usepackage{textcomp}
\usepackage{xcolor}
\def\BibTeX{{\rm B\kern-.05em{\sc i\kern-.025em b}\kern-.08em
    T\kern-.1667em\lower.7ex\hbox{E}\kern-.125emX}}

\begin{document}

\title{Motion Planning\\
}

\author{%%%% author names
    \IEEEauthorblockN{Sandeep Chintada}% first author
    \IEEEauthorblockA{A59015527}
    \IEEEauthorblockA{dchintada@ucsd.edu}
}

\maketitle

\begin{abstract}
    This project aims to compare the performance of search-based and sampling-based motion planning algorithms in 3D Euclidean space. The project will
     involve testing the algorithms in a range of challenging scenarios to assess their effectiveness in handling various types of obstacles.
    
\end{abstract}

\begin{IEEEkeywords}
   Motion Planning, $A^{\ast}$, RRT, Deterministic Shortest Path
\end{IEEEkeywords}
\section{Introduction}
Path planning or motion planning is essential in robotics to determine the optimal path for achieving objectives or reaching target locations. 
It involves generating a sequence of safe and efficient actions to navigate the environment. For instance, self-driving cars use 
path planning to find the best route, while surgical robots rely on it to perform tasks safely and effectively. 

In Project 1, we employed a dynamic programming approach to devise a path to the goal while adhering to the specified constraints. 
However, it became evident that as the constraints increased, the state space expanded exponentially, rendering the planning process intractable 
in complex environments. This was primarily due to the fact that the dynamic programming approach required computing the optimal path from every 
individual location to the goal.

To address this challenge, alternative approaches were formulated and the efficacy of search-based motion planning algorithms, 
such as A*, and sampling-based motion planning algorithms, like RRT was discovered.These algorithms offer a viable solution with significantly reduced computational complexity. 

The $A^\ast$ algorithm stands out for its capability in efficiently exploring discrete state spaces, making it particularly well suited for autonomous navigation and in other
applications where a precise path must be planned, for example, in surgical robotics. 
Sampling-based motion planning algorithms, such as RRT, are effective in high-dimensional configuration spaces with obstacles. 
This method of sampling enables the robot/agent to rapidly explore uncharted regions of the configuration space, which is essential
for navigation in complex and dynamic environments. 
 

\end{document}
